\chapter[Biological motivation]{Biological motivation and background}
\label{kap:biological_motivation}

In this chapter we will introduce the terms and concepts of biology, genetics and computational biology that 
are commonly used through this thesis. 
We will explain phylogenetic trees, their significance in evolutionary biology and 
their connection to the Hyb-Seq protocol and probes. 

\section{Terminology overview, basic terms and concepts}

\subsection{Nucleic acids}
Nucleic acids carry the genetic information of all known living things. Two nucleic acids are called RNA 
(ribonucleic acid) and DNA (deoxyribonucleic acid). They both consist of sequence of nucleotides 
- monomers that are made of a pentose - a sugar with 5 carbons (ribose in RNA, deoxyribose in DNA), a 
phosphate group, and a nitrogenous-base. The five most common bases are cytosine (C), guanine (G), 
adenine (A), thymine (T) and uracil (U). RNA contains C, G, A, U and DNA contains C, G, A, T. These bases 
create hydrogen bonds between each other as follows: C-G, A-T in DNA, and C-G, A-U in RNA. 
RNA is usually single stranded and the bases pair with each other within the same strand, creating 3D-structures. 
Most DNAs is a double helix - it has two complementary strands that pair with each other. 
\cite{cellbiology}

\subsection{Genes}
A gene is a basic unit of heredity, a region in DNA that encodes some function, usually a protein. Several genes 
can encode a single trait or one gene can encode multiple traits. A position, or a place of a gene in DNA is called 
locus (plur. loci). Homologous genes are genes that share a common ancestor. More specificaly, homologous sequences are 
called orthologs, if two copies of the same gene are in two different species and they are called paralogs, if the gene was 
duplicated within the same genome. During evolution, orthologs retain the same function while paralogs (or one of them) can 
can gain new functions. 
Sequences of DNA that are converted into mature mRNA are called exons. The sequences that are between exons are called introns. 
Introns do not code proteins and their sequences can change frequently over time. Exons, on the other hand, are much more 
conserved. 
\cite{cellbiology}

\subsection{Transcription, translation}%central dogma of molecular biology?
When making proteins, regions of DNA are transcribed into a shorter RNA that is complementary to the original DNA sequence. 
This RNA is called messenger RNA or mRNA. mRNA is then translated into proteins. The complete set of all mRNAs from a 
cell or a population of cells is called transcriptome. The transcriptome represents all genes that are being actively expressed. 
%Central dogma of molecular biology is a description of the flow of genetic information. 

Analogicaly, the DNA from mitochondria is called mitochondrion and the DNA from chloroplast is a plastome. 
\cite{cellbiology}
%\section{Construction of phylogenetic trees}

%Here we will describe why is a construction of a phylogenetic tree important for the study of evolution. We will also look 
%into how the tree can be made from the data we have. 

\subsection{Phylogenetic trees}
In biology, the study of evolutionary history amongst organisms, species, populations, etc. is called phylogenetics. Heritable traits 
are evaluated to determine a phylogenetic relationship. Earlier, only morphologic traits could be used, but nowadays, DNA sequences 
or other genetic characteristics are also a valuable genetic markers used in phylogeny. 
\cite{phylogenetics}

A phylogenetic tree is a representation of such relationships. It's a branching diagram, where the taxa that are joined together have 
descended from a common ancestor. 

In this thesis, we tried to find genetic markers which can be used to infer relations in phylogeny of a group of plants. 
\cite{phylogenetic_tree}
