\chapter[Hyb-Seq Protocol]{Hyb-Seq Protocol}
\label{kap:hyb_seq}

In this chapter we will describe in detail the Hyb-Seq Protocol and next generation sequencing, their pros 
and cons. We will address their position in the process of creating a phylogenetic tree and their connection 
to the Sondovac. 

\section{Orthologous genes}
Orthologous genes are fundamental for phylogeny, since they can be used as markers. Selecting effective orthologous genes 
is difficult as gene duplication and deletion is making it hard to tell orthologs from paralogs. Single-copy paralogs that has 
undergone lineage-specific changes can be mistaken for orhologs. 
%(Philippe et al. 2005)
Distinguishing orthologs from paralogs is especially difficult in angiosperms, where polyploidization is a common occurence. 
%(Zhang, 2003; Kellis et al., 2004; Dehal & Boore, 2005; Soltis et al., 2009; Zhou et al., 2010)
%Detecting orthologs? (Tatusov et al. 2003).

\section{Low-copy nuclear genes}
%What are lCNs
Low-copy nuclear genes are genes from nucleus that can be found in the genome in few or single copies. 
%Why are they important in phylogeny
Highly conserved orthologous low-copy nuclear genes have found their use as a source of phylogenetic information. They proved to be effective markers to 
track organismal evolution. 
%zang
%Why nuclear
Most of the protein-coding cell genes can be found in nucleus. Compared to genes from other organelles, nuclear genes from eukaryotic organisms consist 
from several chromosomes. Because of this, genes of eucaryotic organisms are evolutionary unlinked; either on different chromosomes or sufficiently far apart. 
%wang

%How are they related to HybSeq
Finding low-copy nuclear genes has been constrained by bechnical limitations. High sequencing throughput of current platforms, such as Illumina, combined with 
target enrichment enables us to sequence large amounts of low-copy nuclear loci effectively. 
%Hyb-Seq: Combining target enrichment and genome skimming for plant phylogenomics1

\section{Target enrichment}
Hyb-seq protocol is the combination of target enrichment and genome skimming. It enables data collection for low-copy nuclear genes and high-copy genomic targets for 
evolution studies and plant systematics. In hyb-seq, suitable probes are created to serve in target enrichment. %Some other wording?
The principle of target enrichment is selectively finding regions of interest in a genome before sequencing, thus making the sequencing process more effective. %https://www.nature.com/articles/nchembio0705-63
%Maybe write about different methods here?

\cite{sang2002utility}

\section{Next generation sequencing}

\section{Hyb-Seq Protocol}


