\chapter[Practical work and results]{Practical work and results}
\label{kap:results}

In this chapter, we will describe the practical work and process of creating suitable probes. We will take a look at parameters and specifications. 
Lastly, we will summarize the results of the thesis and describe the final data. 

\section{Input data}
We worked with two different sets of input data. These data shared most of the files, differing only in genome. 
%Add some statistics here. #of genes, etc.

The first set used the following data: 


\begin{enumerate}
\item Transcriptome: Alyssum alyssoides
\item Genome: Odontarrhena tortuosa
\item Mitochondrion: Arabidopsis thaliana
\item Chloroplast: Arabidopsis thaliana
\end{enumerate}

The second set used the following data: 


\begin{enumerate}
\item Transcriptome: Alyssum alyssoides
\item Genome: Alyssum gmelinii
\item Mitochondrion: Arabidopsis thaliana
\item Chloroplast: Arabidopsis thaliana
\end{enumerate}

\subsection{Nickel and metallic genes}
We were especially interested in nickel and metallic genes that might be present in these genomes. Therefore, we matched the possible 
probes against nickel and metallic genes and primarily picked those with the greatest similarity. 

%Where do these data come from? Add some statistics

\section{Process overview}
After several tries, it was estabilished that we will make probes from two sets of data to cover broader spectrum of plants they can 
be used for. We ran the Sondovac script for each set and got two resulting sets of possible probes. From these possible probes, we 
proceeded to pick those that fit the criteria the best. Consecutively, the probes have to make up to $1,000,000$ base pairs, due to 
restrictions of the probe-making company. 
To pick suitable probes, we matched each data set to nickel and metal genes and chose those that aligned with them. We matched the 
data sets against each other and picked probes by similarity. We chose only one of the probes that matched. The other one 
wasn't included in the final probes to avoid duplicates. Finally, the rest was picked so they would make up the $1,000,000$ bases to 
make the most of the space. 

%add picture/graph here

\section{Using the pipeline}
The possible probes were produced by running the script Sondovac on two different data sets. We then picked suitable probes from 
the resulting probes. 

%Sondovac parameters
We set the minimal total locus length on $360$ for both datasets - the least possible minimal total locus length - so we could get 
the largest amount of possible probes to pick from. 


\section{Probe picker}

\section{Resulting data}

%Spravila som nejaké grafy
