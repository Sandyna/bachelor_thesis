\chapter[Practical work and results]{Practical work and results}
\label{kap:results}

In this chapter, we will describe the practical work and process of creating suitable probes. We will take a look at parameters and specifications. 
Lastly, we will summarize the results of the thesis and describe the final data. 

\section{Input data}
We worked with two different sets of input data. These data shared most of the files, differing only in genome. 
%Add some statistics here. #of genes, etc.

The first set used the following data: 


\begin{enumerate}
\item Transcriptome: Alyssum alyssoides
\item Genome: Odontarrhena tortuosa
\item Mitochondrion: Arabidopsis thaliana
\item Chloroplast: Arabidopsis thaliana
\end{enumerate}

The second set used the following data: 


\begin{enumerate}
\item Transcriptome: Alyssum alyssoides
\item Genome: Alyssum gmelinii
\item Mitochondrion: Arabidopsis thaliana
\item Chloroplast: Arabidopsis thaliana
\end{enumerate}

\subsection{Nickel and metallic genes}
We were especially interested in nickel and metallic genes that might be present in these genomes. Therefore, we matched the possible 
probes against nickel and metallic genes and primarily picked those with the greatest similarity. 

%Where do these data come from? Add some statistics

\section{Autometed selection of minimum total locus length}

%Definition of the problem goes here
When making probes, we usually want to get exact number of bases, because the created probes are later passed to another 
company, such as MYcroarray (USA) \cite{mycroarray} that will create the actual physical probes from them. These companies 
usually charge for processing of a precise number of bases. To save money, it is best to fill this number to the brim. 
Therefore, when we have several sequences available, we want to select such sequences, that the sum of thei lenghts makes the 
highest number possible, but does not exceed the given limit, let us say $1,000,000$ of base pairs. 

Aside from this, we often have other requirements on the selected sequences, usually based on biological evidence. There might be 
sequences we do not want in our selection, or even sequences we might want to add. 

This selection is usually done by hand or a series of programmes. It would be beneficial, if this part too, was automated. 

We want to create a script, that would work with the second part of sondovac and would be able to select the best possible 
sequences to fill up the limit. This script would take into account any sequences we want to keep, or those we do not want in 
the result. 

We need specific genes - responsible for nickel and metal binding. If these genes are present among our sequences, they have 
to be in the result. We will take the largest possible output file from sondovac\_part\_b: the one with the lowest minimum total locus length. 
From this file, we will select the required genes. Any additional probes will be chosen by a combination of greedy algorithm and a knapsack. 


\section{Process overview}
After several tries, it was estabilished that we will make probes from two sets of data to cover broader spectrum of plants they can 
be used for. We ran the Sondovac script for each set and got two resulting sets of possible probes. From these possible probes, we 
proceeded to pick those that fit the criteria the best. Consecutively, the probes have to make up to $1,000,000$ base pairs, due to 
restrictions of the probe-making company. 
To pick suitable probes, we matched each data set to nickel and metal genes and chose those that aligned with them. We matched the 
data sets against each other and picked probes by similarity. We chose only one of the probes that matched. The other one 
wasn't included in the final probes to avoid duplicates. Finally, the rest was picked so they would make up the $1,000,000$ bases to 
make the most of the space. 

%add picture/graph here

\section{Using the pipeline}
The possible probes were produced by running the script Sondovac on two different data sets. We then picked suitable probes from 
the resulting probes. 

%Sondovac parameters
We set the minimal total locus length on $360$ for both datasets - the least possible minimal total locus length - so we could get 
the largest amount of possible probes to pick from. 


\section{Probe picker}
Probe picker is a script that we coded to help with picking the final probes. It is coded in python. The probe picker requires 
a list of all possible probes to pick from as an input. It can be also given a list of probes we definitely want in the result - such as the 
ones that aligned with the nickel or metallic genes or intersection of the two genomes - and a list of probes we do not want in 
the result - such as duplicates or sequences that are too similar to each other, for 
example the second of the pair of aligned genes from the two genomes we had. 

%Describe the workflow

\subsection{Picker for genes to keep and remove}
We coded additional script that creates the list of probes we want to have in the result and those we want to remove. 

\section{Resulting data}

%Spravila som nejaké grafy
