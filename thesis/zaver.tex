\chapter*{Záver}  % chapter* je necislovana kapitola
\addcontentsline{toc}{chapter}{Záver} % rucne pridanie do obsahu
\markboth{Záver}{Záver} % vyriesenie hlaviciek

%Na záver už len odporúčania k samotnej kapitole Záver v bakalárskej
%práci podľa smernice \cite{smernica}:  \glqq{}V závere je potrebné v
%stručnosti zhrnúť dosiahnuté výsledky vo vzťahu k stanoveným
%cieľom. Rozsah záveru je minimálne dve strany. Záver ako kapitola sa
%nečísluje.\grqq{}

%Všimnite si správne písanie slovenských úvodzoviek okolo
%predchádzajúceho citátu, ktoré sme dosiahli príkazmi \verb'\glqq' a
%\verb'\grqq'.

%V informatických prácach niekedy býva záver kratší ako dve strany, ale
%stále by to mal byť rozumne dlhý text, v rozsahu aspoň jednej strany.
%Okrem dosiahnutých cieľov sa zvyknú rozoberať aj otvorené problémy a
%námety na ďalšiu prácu v oblasti.

%Abstrakt, úvod a záver práce obsahujú podobné informácie. Abstrakt je
%kratší text, ktorý má pomôcť čitateľovi sa rozhodnúť, či vôbec prácu
%chce čítať. Úvod má umožniť zorientovať sa v práci skôr než ju začne
%čítať a záver sumarizuje najdôležitejšie veci po tom, ako prácu
%prečítal, môže sa teda viac zamerať na detaily a využívať pojmy
%zavedené v práci.

In this work, we reviewed basic biological knowledge and principles necessary for understanding this thesis. 
We described a Hyb-seq protocol; the combination of target enrichment and genome skimming which is an approach for determining 
effective genetic markers. We described the significance of orthologous low-copy nuclear genes in phylogeny and explained why they make good 
markers. 

We took a look at Sondovač -- an automated and interactive script for finding orthologous low-copy nuclear probes, which can be later used in 
target enrichment and phylogeny studies. We described in detail each step of the Sondovač pipeline and the software that is used by the script. 

We ran the Sondovač script with two sets of plant data from the family brassicaceae; they both had a transcriptome data from alyssum alyssoides, a mitochondrion and a chloroplast data 
from arabidopsis thaliana and the only difference was the genome skim data. One genome was from odontarrhena tortuosa and the other one was from alyssum gmelinii. 
We successfully went through all three parts of the Sondovač script, including two parts of freeware Sondovač script and the intermediate payware Geneious part, which needed 
a manual input from the first part of Sondovač. We created the largest number of probes possible by using the lowest minimum total locus length as a parameter for Sondovač. 

We had to pick from these probes so their lengths would make a sum closest to $1,000,000$ base pairs. This restriction was due to the requirements of the company that would synthesize
the probes. 
Additionally, we preferentially wanted those probes that aligned with a metal or nickel reference --a reference for genes responsible for metal or nickel binding in plants. We also preffered the 
probes that were present in both of the data sets -- an intersection. 

We aligned the probes with references and both sets of data with LAST. We then coded a filter in python, that would select the probes that we had a priority in the final set of probes and make a
list of probes, for example duplicates or probes that are too similar, which we don't want in the final set of probes. Since there wasn't any obvious leap in numbers of aligned probes based on the 
percentage of the probe's length aligned to the reference nor its similarity, we decided to take the probes in order based on the decreasing similarity. 
Unfortunately, we make an error in this program and accidentaly exchanged the two lists. This resulted into the probes we wanted to keep to be removed and vice versa. 

We created a python script that would pick the probes so they would meet the $1,000,000$ base pairs requirement. The script would take the list of probes to keep and the list of probes to remove as an 
input. The program used a greedy approach to fill up and later dynamic programming to meet the target number of base pairs more precisely. We successfully hit exactly $1,000,000$ base pairs as required. 

However, after sending the final probes for syntesize, it was discovered that many of them contained masked nucleotides, some of them as much as $20\%$ which prevented a successfull syntesis. Therefore, we 
replaced any probes that contained more than $2.5\%$ of degenerated bases by other sequences from the possible probes. 
This could have been caused by the previously mentioned error or by the lower quality of some sequences we got as an input.

Even though the probes were created, it is not certain that they will be of any use. There could be problem with the cross-species amplification, since the best probes from the intersection were removed. 

In the future, som improvements could be made. 
One of the main problems of the Sondovač script is the intermediate part where we have to manually input data from Sondovač part a into Geneious and then use the output 
in the second part. 
It is planned to replace Geneious by another command line tool that would enable full automatization of Sondovač. 

Another possible future work could include the scripts we coded to pick target number of probes. These scripts are not a part of the Sondovač's pipeline and the input has to be given to them manually. Since the data to create a list of probes to keep and remove have specific requirements for every dataset, the pipeline for their automatic selection is not useable in an automated selection of probes. 

However, the Probe Picker script that takes a list of probes to keep and remove as an input and outputs a list of probes that have a certaing target number of bases could be incorporated 
into the Sondovač pipeline. The script would have to be changed to take user input or input parameters instead of parameters hard-coded into the script. BASH could be used to make the script 
a part of the Sondovač pipeline. Some changes would have to be made to the Sondovač script itself to include this script in the pipeline. 

